
In this study, a comprehensive exploration of protein 
sequences within the Plantae kingdom is conducted, with 
particular focus on emerging patterns and relationships 
based on KOG category annotations. The results reveal 
insights into the diversity and functional adaptations of 
plant proteomes, shedding light on their adaptive history 
and potential ecological implications. This section 
synthesizes the current findings and provides a broader 
context for their importance.


\subsection{KOG Category Counts Analysis}
\label{sec:conclusion.kogcount}

The analysis of KOG category counts across the 114 plant 
proteomes reveals notable patterns of protein functional 
distribution. Proteomes show varying degrees of 
representation, with \emph{A. thaliana} standing out as 
the best-represented proteome. Other plants, although under 
represented, have an important role in human nutrition or 
health industries. An example is the unicellular green 
alga \emph{D. salina}, whose 
biomass or its extracts have been shown to produce 
positive effects on the treatment of cardiovascular 
diseases and cancer, as well as immunomodulatory and 
anti-inflammatory properties~\citep{hyrslova2022}. 
Carrot (\emph{D. carota}) is another plant 
with a low representation in this study but with an 
outstanding importance due to its antioxidant, 
anti-inflammatory, plasma lipid modification, and 
antitumor properties, which can help reduce the risk of 
cancer and cardiovascular diseases~\citep{ahmad2019}. The 
least represented plant \emph{A. trichopoda} has a rich 
phytochemical landscape that may have potential applications 
in the pharmaceutical and biotechnology industries, due to 
its diverse glycosylated flavonoids, anthocyanins, and 
proanthocyanidins~\citep{wu2019}. 

Considering the normalized count results shown in 
Figure~\ref{fig:EMAO}, there are many aspects to consider.
First, Eudicots and Monocots, the two most closely 
related groups, shows some differences in functional 
categories related to metabolism. Namely, a rupture 
in the trend can be seen on 
the categories of transport and metabolism of inorganic ions 
(P), amino acids (E), and carbohydrates (G). These 
differences could be associated to core differences between 
the two groups at the physiological level, especially 
regarding nutrient acquisition: Eudicots have one main root 
(taproot) where other smaller roots branch off, while 
Monocots have more fibrous roots webbing off in different 
directions~\citep{freschet2021}. Furthermore, evidence 
indicates that metabolic pathways related to nitrogen 
uptake, carbohydrate accumulation, and metal 
transport show differences between Eudicots and 
Monocots~\citep{yang2020,tian2016}.
Secondly, algae display a lack of representation on some 
functional categories, most of them related to cellular 
processes and signaling. On the other hand, the category 
Chromatin structure and dynamics (B) stands out among other 
categories and biological groups. This behavior can also 
be noted on Figure~\ref{fig:clustermap} for the cluster
related to most algae. Although some evidence has been found on 
the role of epigenetics modifications on microalgae and how 
they adapt to environmental conditions, no clear 
In this regard, evidence shows that the genome of \emph{C. 
reinhardtii} has an unusual pattern of methylation when compared 
to other plants or animals~\citep{bacova2020}. Furthermore, 
\cite{vigneau2021} explain that DNA methylation seems to govern 
speciation in plants: in green algae, it occurs in gene-poor 
regions, while in bryophytes marks genes and is crucial for the 
life cycle of the plant. On the other hand, DNA methylation for 
\emph{Arabidopsis} and Angiosperms occurs by silencing some 
gametophyte-specific genes.

Overall, the varying 
representation of KOG categories among different taxonomic 
groups, such as Eudicots, Monocots, and Algae, provides 
insights into potential differences in functional roles 
and adaptations.


\subsection{Hierarchical Classification Results}
\label{sec:conclusion.hierarchy}

The hierarchical organization of plant species based on 
KOG frequency counts highlights clusters that align with 
both known taxonomic classifications and adaptive 
relationships. The presence of distinct clusters, 
particularly the \emph{Brassicaceae} cluster and the 
algae-rich group, reflects the diversity within the 
Plantae kingdom and the similarities on a broad approach to
hierarchical classification in comparison to the taxonomic tree.
Important to note is the fact that the 5 non-algae plants related 
to the latter group have some agricultural importance: Carrot 
(\emph{D. carota}), wheat (\emph{T. aestivum}), lettuce 
(\emph{L. sativa}), pineapple (\emph{A. comosus}), and 
prince's-feather (\emph{A. hypochondriacus}).

The emergence of few well-defined mini-clusters, including 
\emph{A. thaliana}, \emph{Citrus}, and \emph{Sorghum}, 
calls attention to the shared functional roles among 
species from the same genus. The placement of 
\emph{A. trichopoda} and the plants from the Monocot and 
Other biological groups is a matter worth a closer look, 
since no conclusion about their classification can be 
established based on these results.


\subsection{Comparison between Gene Ontology and Phytozome}
\label{sec:conclusion.comparison}

The comparison between KOG-based and GO-based 
classifications revealed a high degree of concordance, 
with a Spearman's rank correlation coefficient of 0.98. 
This suggests the robustness of the KOG annotations and 
their alignment with GO annotations in characterizing 
protein functions. The minor differences observed in the 
rankings of certain categories stress the complementary 
nature of these databases in providing a comprehensive 
understanding of protein functionality. Differential 
enrichment of certain categories between the databases 
hints at distinct emphasis in annotation curation.


\subsection{Conclusion}
\label{sec:conclusion.conclusion}

This study contributes to our understanding of plant 
proteomics and adaptive biology in several ways. The 
analysis of KOG category counts provides insights into 
the functional diversity and adaptations of plant 
proteomes, highlighting key categories involved in 
essential biological processes. The hierarchical 
classification offers a visualization of adaptation 
relationships and clusters that align with known 
taxonomic groups. These findings offer valuable 
resources for future research, including investigations 
into convergent evolution, functional adaptations, and the 
molecular mechanisms governing speciation.

These results contribute to the broader field of 
bioinformatics and proteomics, offering new perspectives on 
the functional diversity and evolutionary relationships 
among plant species. This study lays the foundation for 
continued exploration of plant proteomics and its 
implications for understanding plant-environment 
interactions and adaptations.

In conclusion, the comprehensive analysis of protein 
sequences within the Plantae kingdom using KOGs has 
unveiled intricate patterns of functional distribution, 
evolutionary relationships, and shared annotations with 
the GO database. The integration of bioinformatics tools, 
hierarchical classification, and statistical analyses has 
provided a deeper understanding of plant biology and evolution. 
This study paves the way for further research and 
applications in molecular biology, bioinformatics, and 
proteomics, ultimately enhancing our knowledge of 
plant-environment interactions and adaptive responses to 
changing conditions.
