In this study, a comprehensive exploration of protein 
sequences within the Plantae kingdom is conducted, with 
particular focus on emerging patterns and relationships 
based on KOG category annotations. The results reveal 
insights into the diversity and functional adaptations of 
plant proteomes, shedding light on their adaptive history 
and potential ecological implications. This section 
synthesizes the current findings and provides a broader 
context for their importance.


\subsection{KOG Category Counts Analysis}
\label{sec:conclusion.kogcount}

The analysis of KOG category counts across the 114 plant 
proteomes reveals notable patterns of protein functional 
distribution. Proteomes show varying degrees of 
representation, with \emph{A. thaliana} standing out as 
the best-represented proteome. However, it's crucial to 
note that less-represented plants may hold significant 
importance in various industries and human health. 

For instance, the unicellular green 
alga \emph{D. salina} is underrepresented but has shown 
promise in treatments for cardiovascular diseases and 
cancer, as well as immunomodulatory and anti-inflammatory 
properties~\citep{hyrslova2022}. 
Carrot (\emph{D. carota}) is another plant 
with low representation in this study but has significant 
relevance due to its antioxidant, anti-inflammatory, and 
potential anticancer properties, which can contribute to 
reducing the risk of cancer and cardiovascular 
diseases~\citep{ahmad2019}. The 
least represented plant \emph{A. trichopoda} has a rich 
phytochemical landscape with potential applications in 
pharmaceutical and biotechnology industries, thanks to 
its diverse glycosylated flavonoids, anthocyanins, and 
proanthocyanidins~\citep{wu2019}. 

Considering the normalized count results shown in 
Figure~\ref{fig:EMAO}, several insights emerge.
First, Eudicots and Monocots, the two most closely 
related groups, exhibit differences in functional 
categories related to metabolism. Namely, a rupture 
in the trend can be seen on 
the categories of transport and metabolism of inorganic ions 
(P), amino acids (E), and carbohydrates (G). Notably, 
differences in nutrient acquisition strategies may 
underlie these distinctions: Eudicots have one main root 
(taproot) where other smaller roots branch off, while 
Monocots have more fibrous roots webbing off in different 
directions~\citep{freschet2021}. Studies have also highlighted 
variations in metabolic pathways related to nitrogen uptake, 
carbohydrate accumulation, and metal transport between 
Eudicots and Monocots~\citep{yang2020,tian2016}.

Secondly, algae display a lack of representation on some 
functional categories, most of them related to cellular 
processes and signaling. On the other hand, the category 
Chromatin structure and dynamics (B) stands out among other 
categories and biological groups. This behavior can also 
be noted on Figure~\ref{fig:clustermap} for the cluster
related to most algae.
This divergence in functional categories suggests unique 
genomic and epigenetic features in these organisms, such 
as unusual patterns of DNA methylation in some algae 
species~\citep{bacova2020}. Furthermore, 
\cite{vigneau2021} explain that DNA methylation seems to govern 
speciation in plants: in green algae, it occurs in gene-poor 
regions, while in bryophytes marks genes and is crucial for the 
life cycle of the plant. On the other hand, DNA methylation for 
\emph{Arabidopsis} and Angiosperms occurs by silencing some 
gametophyte-specific genes.

The varying 
representation of KOG categories among different taxonomic 
groups, such as Eudicots, Monocots, and Algae, provides 
insights into potential differences in functional roles 
and adaptations in plants.


\subsection{Hierarchical Classification Results}
\label{sec:conclusion.hierarchy}

The hierarchical organization of plant species based on 
KOG frequency counts highlights clusters that align with 
both known taxonomic classifications and adaptive 
relationships. The presence of distinct clusters, 
particularly the \emph{Brassicaceae} cluster and the 
algae-rich group, reflects the diversity within the 
Plantae kingdom and shows similarities with taxonomic trees.

It is noteworthy that the cluster comprising algae also 
includes several non-algae plants with agricultural 
importance, such as  Carrot 
(\emph{D. carota}), wheat (\emph{T. aestivum}), lettuce 
(\emph{L. sativa}), pineapple (\emph{A. comosus}), and 
prince's-feather (\emph{A. hypochondriacus}).

Some well-defined mini-clusters emerge, including 
\emph{A. thaliana}, \emph{Citrus}, and \emph{Sorghum}, 
indicating shared functional roles among species of the 
same genus. However, the placement of 
\emph{A. trichopoda} and the plants from the 
\emph{Monocot} and \emph{Other} biological groups 
requires further investigation, and no definitive 
conclusions about their classification can be drawn 
based on these results.
Our approach differs from other methods found in the 
literature, such as PlantTribes2, The Plant Orthology 
Browser, or InParanoiDB 9, as it focuses on functional 
category counts rather than sequence 
alignment~\citep{wafula2023,tulpan2017, persson2023}. 
While these approaches provide valuable insights, our 
goal is to assess the importance of functional 
annotations in understanding adaptive processes.


\subsection{Comparison between Gene Ontology and Phytozome}
\label{sec:conclusion.comparison}

The comparison between Phytozome-based and GO-based 
classifications revealed a high degree of concordance, 
with a Spearman's rank correlation coefficient of 0.98. 
This confirms the robustness of the KOG annotations and 
their alignment with GO annotations in characterizing 
protein functions~\citep{tatusov2003}. 
The minor differences observed in the 
rankings of certain categories stress the complementary 
nature of these databases in providing a comprehensive 
understanding of protein functionality. Differential 
enrichment of certain categories between the databases 
hints at distinct emphasis in annotation curation.

Regarding the inference of species relationships, the 
approach presented in this study also differs from the one 
presented by~\cite{wafula2023} in PlantTribes2, where 
functional annotations are extracted from Gene Ontology, 
InterPro/Pfam protein domains, The Arabidopsis Information 
Resource (TAIR), UniProtKB/TrEMBL, and UniProtKB/Swiss-Prot.
However, the approach presented in this study differs from 
those of PlantTribes2, The Plant Orthology Browser, and 
InparanoiDB because it uses the counting of annotated proteins 
for each functional category in KOG, instead of directly using 
some form of sequence alignment. It does not mean that the 
presented approach uses no alignment whatsoever, but it is only 
a preprocessing that helps in retrieving the mentioned counts.



\subsection{Conclusion}
\label{sec:conclusion.conclusion}

The comprehensive analysis of protein 
sequences within the Plantae kingdom using KOGs has 
unveiled intricate patterns of functional distribution, 
evolutionary relationships, and shared annotations with 
the GO database. Namely, This study presents 
how the functional annotations vary among the defined plant 
clusters. These averaged distributions of category counts 
can serve as an initial criterion for classifying unobserved 
organisms based on the similarity of their category counts to 
established clusters. Furthermore, the applied methodology 
for identifying these functional annotation distributions 
can be extended to more organisms or even applied again 
from scratch for other life kingdoms, where a 
functional criterion for classification could elucidate 
additional relationships between organisms. A broad coverage of 
the studied kingdom is highly encouraged.

Moreover, the clustering patterns revealed in this study, 
some of which align with phylogenetic groups, emphasize 
the potential connection between functional adaptations 
and the proteome's response to environmental cues. For 
instance, the \emph{Brassicaceae} or the algae-rich 
cluster, as observed in Figure~\href{fig:clustermap}, 
are very similar to their reference counterparts. However, 
this study does not assess the causality of that connection.
Overall, the integration of bioinformatics tools, 
hierarchical classification, and statistical analyses provides 
a deeper understanding of plant biology and adaptive processes. 
This study paves the way for further research and 
applications in molecular biology, bioinformatics, and 
proteomics, ultimately enhancing our knowledge of 
plant-environment interactions and adaptive responses to 
changing conditions.
