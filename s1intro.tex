%% Why bother?
Proteins, as the building blocks of life on Earth, play a crucial 
role in cellular maintenance, replication, and defense across all 
living organisms~\citep{kaur2022}. They are involved in diverse 
cellular functions, including signal transduction, regulation 
of metabolic pathways, and responses to environmental 
stimuli~\citep{zhang2010}.

The study of proteins and their functions is a central focus in 
molecular biology and bioinformatics research. Understanding the 
biological function of proteins is essential for comprehending 
the genetics of an organism~\citep{silva2020}. Among living 
organisms, plants hold particular significance due to their 
vital roles in photosynthesis and the establishment of various 
ecosystems. The Plantae kingdom, encompassing a vast array of 
species, exhibits unparalleled diversity in protein sequences 
that have evolved over millions of years~\citep{chaudhary2019}. 
This diversity arises from complex gene regulation processes, 
including co-transcriptional, post-transcriptional, and 
post-translational regulation~\citep{skelly2016}. Analyzing 
patterns and variations within plant protein sequences can 
provide invaluable insights into their evolutionary history 
and adaptations to different ecological niches.

Previous bioinformatics research has highlighted the utility 
of the Clusters of Orthologous Groups (COGs) database for 
protein annotation in prokaryotes and unicellular 
eukaryotes~\citep{tatusov1997,tatusov2001}. An updated version, 
the euKaryotic Orthologous Groups (KOGs), includes annotated 
proteomes for seven eukaryotic model organisms, facilitating 
the comparison of protein sequences across species and 
identification of orthologous genes with common ancestry in 
other eukaryotes~\citep{tatusov2003,yang2023,wangC2023,wangT2023}.

%% What do we do?
This study uses the KOGs framework to analyze 114 plant 
proteomes. 
%% How do we do it?
Reference annotations from \emph{Arabidopsis 
thaliana} in the KOG database are transferred to each 
organism using BLASTP alignments~\citep{camacho2009} 
and counted in relative frequency for comparison. 
Normalized counts adding up to 1 are used also to make 
comparisons among relevant biological categories.
%% what for?
Identifying patterns of functional annotations among 
plant species provides insights into adaptive 
mechanisms to respond to different environmental 
situations.
