%% Why bother?
Proteins, as the building blocks of life on Earth, play a 
crucial role in cellular maintenance, replication, and defense 
across all living organisms~\citep{kaur2022}. They are involved 
in diverse cellular functions, including signal transduction, 
regulation of metabolic pathways, and responses to environmental 
stimuli~\citep{zhang2010}.

The study of proteins and their functions is a central focus in 
molecular biology and bioinformatics research. Understanding the 
biological function of proteins, also known as functional 
annotations, is essential for comprehending 
the genetics of an organism~\citep{silva2020}. Furthermore, these 
functional annotations are useful when computationally improving 
classification or correlation tasks on proteins, especially those 
involving noisy data. 

Among living 
organisms, plants hold particular significance due to their 
vital roles in photosynthesis and the establishment of various 
ecosystems. Moreover, their adaptive mechanisms consist of a 
wide range of molecular processes, many of which still remain 
unidentified. The Plantae kingdom, encompassing a vast array of 
species, exhibits unparalleled diversity in protein sequences 
that have evolved over millions of years~\citep{chaudhary2019}. 
Such diversity arises from complex gene regulation processes 
that define cell response to stimuli or plant adaptation, 
including co-transcriptional, post-transcriptional, and 
post-translational regulation~\citep{skelly2016}. In 
consequence, analyzing 
patterns and variations within plant protein sequences can 
provide invaluable insights into their evolutionary history 
and adaptations to different ecological niches, which could be 
used in current analysis methods to establish 
relationships among plant species.

When assessing adaptive mechanisms in plant cells, the 
identification of proteins or their unique attributes
involves an intricate process due to the complexity of plant 
genomes and their expression products. Although 
computational models enhance the definition of significant elements 
in plant cell biology, this process could potentially lead to an 
extrapolation of overrepresented information from databases, rather 
than capturing the entirety of plant understanding. In order to 
to provide an identification framework for homologous relationships 
between proteins of different organisms, the database of 
% Previous bioinformatics research has highlighted the utility of the 
Clusters of Orthologous Groups (COGs) is
created~\citep{tatusov1997,tatusov2001}. Its goal is to enable 
protein annotations in prokaryotes and unicellular 
eukaryotes. An updated version, 
the euKaryotic Orthologous Groups (KOGs), includes annotated 
proteomes for seven eukaryotic model organisms, facilitating 
the comparison of protein sequences across species and 
identification of orthologous genes with common ancestry in 
other eukaryotes~\citep{tatusov2003,yang2023,wangC2023,wangT2023}.

%% What do we do?
This study uses the KOG framework to analyze 114 plant 
proteomes. 
%% How do we do it?
Reference annotations from \emph{Arabidopsis 
thaliana} in the KOG database are transferred to each 
organism using local alignments, so that patterns in the 
frequency of certain functional categories are elucidated.
The previous process identifies relevant biological clusters of 
plants when analyzed as a hierarchical classification of 
the species.
%% what for?
The identified patterns among functional annotations and 
plant species provide insights into adaptive 
mechanisms to respond to different environmental 
situations. Furthermore, these patterns elucidate the close 
relationship between physiological differences and metabolic 
pathways involved in adaptive processes, which could be 
considered as integrating several levels of so-called 
omic information.
